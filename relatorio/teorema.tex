\makeatletter

\usepackage{amsthm, amsmath, amssymb, bm, mathtools}
\usepackage{enumitem, etoolbox, xpatch}


% tira o ponto do nome do teorema
\AtBeginDocument{\xpatchcmd{\@thm}{\thm@headpunct{.}}{\thm@headpunct{}}{}{}}

% teoremas mais simples
% \newtheorem*{ntheorem}{Teorema.}
\newtheorem*{intheorem}{Teorema Incorreto.}
\newtheorem*{ptheorem}{Teorema?}


% faz label com nome quando não está vazio
\newcommand{\set@label}[2]{
    \ifstrempty{#1}{}{%
        \begingroup%
            \edef\@currentlabel{#2}%
            \label{#1}%
        \endgroup%
    }
}

% prova por casos
\newcounter{cases@cnt}
\newtheoremstyle{case}{0.2em}{0.2em}{}{\parindent}{\itshape}{.}{0.8em}{\thmname{#1}\thmnumber{ #2}\thmnote{: ~#3~}}
\theoremstyle{case}
\newtheorem{case}[cases@cnt]{Caso}
\def\resetCasos{\setcounter{cases@cnt}{0}}

% ambiente base dos teoremas
\newtheoremstyle{theorem@sty}{0.2em}{0.2em}{\itshape}{}{\bfseries}{:}{0.8em}{\thmname{#1}\thmnote{ #3}\thmnumber{.#2}}
\theoremstyle{theorem@sty}
\newtheorem*{theorem@thm}{Teorema}

% contador dos teoremas
\newcounter{theorem@cnt}[subsection]

% ambientes dos teoremas
\newenvironment{theorem}[1][]{%
    \resetCasos%
    \refstepcounter{theorem@cnt}%
    \edef\tmp@val{\thetheorem@cnt}%
    \begin{theorem@thm}[\cur@itemname.\tmp@val]%
        \set@label{#1}{\cur@itemname.\tmp@val}%
}{%
    \end{theorem@thm}%
}
\newenvironment{theorem*}[1][]{%
    \resetCasos%
    \begin{theorem@thm}[\cur@itemname]%
        \set@label{#1}{\cur@itemname}%
}{%
    \end{theorem@thm}%
}

% teorema nomeado
\newenvironment{ntheorem}[2][]{%
    \resetCasos%
    \begin{theorem@thm}[#2]%
        \set@label{#1}{#1}%
}{%
    \end{theorem@thm}%
}

% ambiente para lemas
\newtheorem*{theorem@lemma}{Lema}
\newenvironment{lemma}[1][]{%
    \resetCasos%
    \refstepcounter{theorem@cnt}%
    \edef\tmp@val{\thetheorem@cnt}%
    \begin{theorem@lemma}[\cur@itemname.\tmp@val]%
        \set@label{#1}{\cur@itemname.\tmp@val}%
}{%
    \end{theorem@lemma}%
}

% comentários
\newenvironment{comments}[1][Comentários]{%
    \begin{proof}[#1]%
        \edef\qedsymbol{}%
}{%
    \end{proof}%
}

% objetivo do bloco da prova
\newcommand{\objetivo}[1]{$\left(\text{\textit{#1}}\right)$}

\makeatother
